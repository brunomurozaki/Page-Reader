%% abtex2-modelo-trabalho-academico.tex, v-1.9.2 laurocesar
%% Copyright 2012-2014 by abnTeX2 group at http://abntex2.googlecode.com/ 
%%
%% This work may be distributed and/or modified under the
%% conditions of the LaTeX Project Public License, either version 1.3
%% of this license or (at your option) any later version.
%% The latest version of this license is in
%%   http://www.latex-project.org/lppl.txt
%% and version 1.3 or later is part of all distributions of LaTeX
%% version 2005/12/01 or later.
%%
%% This work has the LPPL maintenance status `maintained'.
%% 
%% The Current Maintainer of this work is the abnTeX2 team, led
%% by Lauro César Araujo. Further information are available on 
%% http://abntex2.googlecode.com/
%%
%% This work consists of the files abntex2-modelo-trabalho-academico.tex,
%% abntex2-modelo-include-comandos and abntex2-modelo-references.bib
%%

% ------------------------------------------------------------------------

\documentclass[
	% -- opções da classe memoir --
	12pt,				% tamanho da fonte
	% openright,			% capítulos começam em pág ímpar (insere página vazia caso preciso)
	oneside,			% para impressão apenas no anverso (apenas frente). Oposto a twoside
	a4paper,			% tamanho do papel. 
	% -- opções da classe abntex2 --
	%chapter=TITLE,		% títulos de capítulos convertidos em letras maiúsculas
	%section=TITLE,		% títulos de seções convertidos em letras maiúsculas
	%subsection=TITLE,	% títulos de subseções convertidos em letras maiúsculas
	%subsubsection=TITLE,% títulos de subsubseções convertidos em letras maiúsculas
	% -- opções do pacote babel --
	english,			% idioma adicional para hifenização
	%french,				% idioma adicional para hifenização
	%spanish,			% idioma adicional para hifenização
	brazil				% o último idioma é o principal do documento
	]{abntex2ppgsi}

% ---
% Pacotes básicos 
% ---
% \usepackage{lmodern}			% Usa a fonte Latin Modern			
% \usepackage[T1]{fontenc}		% Selecao de codigos de fonte.
\usepackage[utf8]{inputenc}		% Codificacao do documento (conversão automática dos acentos)
\usepackage{indentfirst}		% Indenta o primeiro parágrafo de cada seção.
\usepackage{color}				% Controle das cores
\usepackage{graphicx}			% Inclusão de gráficos
\usepackage{microtype} 			% para melhorias de justificação
\usepackage{pdfpages}     %para incluir pdf
\usepackage{algorithm}			%para ilustrações do tipo algoritmo
\usepackage{mdwlist}			%para itens com espaço padrão da abnt
\usepackage[noend]{algpseudocode}			%para ilustrações do tipo algoritmo
		
% ---
% Pacotes adicionais, usados apenas no âmbito do Modelo Canônico do abnteX2
% ---
\usepackage{lipsum}				% para geração de dummy text
% ---

% ---
% Pacotes de citações
% ---
\usepackage[brazilian,hyperpageref]{backref}	 % Paginas com as citações na bibl
\usepackage[alf]{abntex2cite}	% Citações padrão ABNT

% --- 
% CONFIGURAÇÕES DE PACOTES
% --- 

% ---
% Configurações do pacote backref
% Usado sem a opção hyperpageref de backref
\renewcommand{\backrefpagesname}{Citado na(s) página(s):~}
% Texto padrão antes do número das páginas
\renewcommand{\backref}{}
% Define os textos da citação
\renewcommand*{\backrefalt}[4]{
	\ifcase #1 %
		Nenhuma citação no texto.%
	\or
		Citado na página #2.%
	\else
		Citado #1 vezes nas páginas #2.%
	\fi}%
% ---

% ---
% Informações de dados para CAPA e FOLHA DE ROSTO
% ---

%-------------------------------------------------------------------------
\instituicao{
	UNIVERSIDADE DE SÃO PAULO
	\par
	ESCOLA DE ARTES, CIÊNCIAS E HUMANIDADES
	\par
	RESOLUÇÃO DE PROBLEMAS II - ACH 0042}

%-------------------------------------------------------------------------

%-------------------------------------------------------------------------
% Informações sobre o ``título'':
%
% Em maiúscula apenas a primeira letra da sentença (do título), exceto 
% nomes próprios, geográficos, institucionais ou Programas ou Projetos ou 
% siglas, os quais podem ter letras em maiúscula também.
%
% O subtítulo do trabalho é opcional.
% Sem ponto final.
%-------------------------------------------------------------------------
\titulo{Acessibilidade de paginas WEB para usuários com deficiências visuais e auditivas}

%-------------------------------------------------------------------------
% Informações sobre o ``autor'':
%
% Todas as letras em maiúsculas.
% Nome completo.
% Sem ponto final.
%-------------------------------------------------------------------------
\autor{\uppercase{Bruno Impossinato Murozaki} 
		\and 
		\uppercase{Rodrigo Guerra}}

%-------------------------------------------------------------------------
% Informações sobre o ``local'':
%
% Não incluir o ``estado''.
% Sem ponto final.
%-------------------------------------------------------------------------
\local{São Paulo}

%-------------------------------------------------------------------------
% Informações sobre a ``data'':
%
% Colocar o ano do depósito (ou seja, o ano da entrega). 
%
% Não incluir o dia, nem o mês.
% Sem ponto final.
%-------------------------------------------------------------------------
\data{2016}

%-------------------------------------------------------------------------
% Informações sobre o ``Orientador'':
%
% Se for uma professora, trocar por ``Profa. Dra.''
% Nome completo.
% Sem ponto final.
%-------------------------------------------------------------------------
\orientador{Profa. Dra. Sarajane Marques Peres}
%-------------------------------------------------------------------------
\tipotrabalho{Disciplina Resolução de Problemas II - ACH 0042}

\preambulo{
%-------------------------------------------------------------------------
% Versão original \newline \newline \newline
%-------------------------------------------------------------------------
 Projeto de pesquisa apresentado à Escola de Artes, Ciências e Humanidades da Universidade de São Paulo como requisito para a disciplina Resolução de Problemas II - ACH 0042.
}
%-------------------------------------------------------------------------
% Configurações de aparência do PDF final

% alterando o aspecto da cor azul
\definecolor{blue}{RGB}{41,5,195}

% informações do PDF
\makeatletter
\hypersetup{
     	%pagebackref=true,
		pdftitle={\@title}, 
		pdfauthor={\@author},
    	pdfsubject={\imprimirpreambulo},
	    pdfcreator={LaTeX com abnTeX2},
		pdfkeywords={abnt}{latex}{abntex}{abntex2}{projeto de pesquisa}{Disciplina Resolução de Problemas II - ACH 0042}{RPII}, 
		colorlinks=true,       		% false: boxed links; true: colored links
    	linkcolor=blue,          	% color of internal links
    	citecolor=blue,        		% color of links to bibliography
    	filecolor=magenta,      		% color of file links
		urlcolor=blue,
		bookmarksdepth=4
}
\makeatother
% --- 

% --- 
% Espaçamentos entre linhas e parágrafos 
% --- 

% O tamanho do parágrafo é dado por:
\setlength{\parindent}{1.25cm}

% Controle do espaçamento entre um parágrafo e outro:
\setlength{\parskip}{0cm}  % tente também \onelineskip
\renewcommand{\baselinestretch}{1.5}

% ---
% compila o indice
% ---
\makeindex
% ---

	% Controlar linhas orfas e viuvas
  \clubpenalty10000
  \widowpenalty10000
  \displaywidowpenalty10000

% ----
% Início do documento
% ----
\begin{document}

% Retira espaço extra obsoleto entre as frases.
\frenchspacing 

% ----------------------------------------------------------
% ELEMENTOS PRÉ-TEXTUAIS
% ----------------------------------------------------------
% \pretextual

% ---
% Capa
% ---
%-------------------------------------------------------------------------
% Informações sobre a ``capa'':
%
% Esta é a ``capa'' principal/oficial do trabalho, a ser impressa apenas 
% para os casos de encadernação simples (ou seja, em ``espiral'' com 
% plástico na frente).
% 
% Não imprimir esta ``capa'' quando houver ``capa dura'' ou ``capa brochura'' 
% em que estas mesmas informações já estão presentes nela.
%
%-------------------------------------------------------------------------
\imprimircapa
% ---

% ---
% Folha de rosto
% (o * indica que haverá a ficha bibliográfica)
% ---
\imprimirfolhaderosto
% ---

% ---
% RESUMOS
% ---

% resumo em português
\setlength{\absparsep}{18pt} % ajusta o espaçamento dos parágrafos do resumo
\begin{resumo}

%-------------------------------------------------------------------------
Dentro do mundo da Internet, muito se discute sobre elementos que tornam a vida do usuário comum mais fácil, com páginas desenhadas de forma acessível. Mas existem ainda problemas relacionados a usuários que possuem inabilidades específicas. Traremos neste relatório as dificuldades e os problemas que pessoas com deficiências visuais e auditivas têm ao acessar o conteudo na WEB, bem como os passos para o desenvolvimento de aplicações que facilitam o uso da internet para estas pessoas.

Palavras-chaves: Internet. Deficiência. Auditiva. Visual. Acesso. Conteúdo. Acessibilidade.
\end{resumo}

% resumo em inglês
%-------------------------------------------------------------------------
% Informações sobre ``resumo em inglês''
% 
% Caso o projeto inteiro seja elaborada no idioma inglês, 
% então o ``Abstract'' vem antes do ``Resumo''.
% 
%-------------------------------------------------------------------------
\begin{resumo}[Abstract]
\begin{otherlanguage*}{english}
%-------------------------------------------------------------------------
Write here the English version of your ``Resumo''. Example text, example text, example text, example text, example text, example text, example text, example text, example text, example text, example text, example text, example text, example text, example text, example text, example text, example text, example text, example text, example text, example text, example text, example text, example text, example text, example text, example text, example text, example text, example text, example text, example text, example text, example text, example text, example text, example text, example text, example text, example text, example text, example text, example text, example text, example text, example text.

Keywords: Keyword1. Keyword2. Keyword3. Etc.
\end{otherlanguage*}
\end{resumo}

% ---
% ---
% inserir lista de figuras
% ---
\pdfbookmark[0]{\listfigurename}{lof}
\listoffigures*
\cleardoublepage
% ---

% ---
% inserir lista de algoritmos
% ---
\pdfbookmark[0]{\listalgorithmname}{loa}
\listofalgorithms
\cleardoublepage

% ---
% inserir lista de tabelas
% ---
\pdfbookmark[0]{\listtablename}{lot}
\listoftables*
\cleardoublepage
% ---

% ---
% inserir lista de abreviaturas e siglas
% ---
%-------------------------------------------------------------------------
% Informações sobre ``Lista de abreviaturas 
% e siglas'': 
%
% Opcional.
% Uma vez que se deseja usar, é necessário manter padrão e consistência no
% trabalho inteiro.
% Se usar: inserir em ordem alfabética.
%
%-------------------------------------------------------------------------
\begin{siglas}
  \item[Sigla/abreviatura 1] Definição da sigla ou da abreviatura por extenso
  \item[Sigla/abreviatura 2] Definição da sigla ou da abreviatura por extenso
  \item[Sigla/abreviatura 3] Definição da sigla ou da abreviatura por extenso
  \item[Sigla/abreviatura 4] Definição da sigla ou da abreviatura por extenso
  \item[Sigla/abreviatura 5] Definição da sigla ou da abreviatura por extenso
  \item[Sigla/abreviatura 6] Definição da sigla ou da abreviatura por extenso
  \item[Sigla/abreviatura 7] Definição da sigla ou da abreviatura por extenso
  \item[Sigla/abreviatura 8] Definição da sigla ou da abreviatura por extenso
  \item[Sigla/abreviatura 9] Definição da sigla ou da abreviatura por extenso
  \item[Sigla/abreviatura 10] Definição da sigla ou da abreviatura por extenso
\end{siglas}
% ---

% ---
% inserir lista de símbolos
% ---
%-------------------------------------------------------------------------
% Informações sobre ``Lista de símbolos'': 
%
% Opcional.
% Uma vez que se deseja usar, é necessário manter padrão e consistência no
% trabalho inteiro.
% Se usar: inserir na ordem em que aparece no texto.
% 
%-------------------------------------------------------------------------
\begin{simbolos}
  \item[$ \Gamma $] Letra grega Gama
  \item[$ \Lambda $] Lambda
  \item[$ \zeta $] Letra grega minúscula zeta
  \item[$ \in $] Pertence
\end{simbolos}
% ---

% ---
% inserir o sumario
% ---
\pdfbookmark[0]{\contentsname}{toc}
\tableofcontents*
\cleardoublepage
% ---

% ----------------------------------------------------------
% ELEMENTOS TEXTUAIS
% ----------------------------------------------------------
\textual



%-------------------------------------------------------------------------
% Informações sobre ``títulos de seções''
% 
% Para todos os títulos (seções, subseções, tabelas, ilustrações, etc):
%
% Em maiúscula apenas a primeira letra da sentença (do título), exceto 
% nomes próprios, geográficos, institucionais ou Programas ou Projetos ou
% siglas, os quais podem ter letras em maiúscula também.
%
%-------------------------------------------------------------------------
\chapter{Introdução}
\label{sec:intro}

% Inicio do trabalho

O estudo de acessibilidade de ferramentas WEB é muito amplo. Existem diversos estudos e padrões definidos dentro da literatura que definem o que é ou não acessível. A entidade reguladora de padrões WEB, a \textit{World Wide WEB Consortium} (W3C) especifica em seu guia diversos itens que facilitam o acesso do usuário, através a Iniciativa de Acessibilidade WEB \citeonline{w3c_wai}. 

Mas há ainda na literatura uma falta de análise mais profunda quanto a questão de acessibilidade para com os usuários com algum tipo de deficiência ou inabilidade específica. A própria W3C, através da mesma Iniciativa de Acessibilidade WEB (WCAG) traz alguns pontos sobre como pessoas com inabilidades utilizam a WEB \citeonline{w3c_wai_disabilites}.

Com o passar dos anos, algumas ferramentas isoladas permitem com que haja a inclusão por parte dos usuários com algumas inabilidades específicas. As ferramentas de leituras de texto (\textit{Text To Speech}) por exemplo, dão ao usuário a chance de reproduzir o texto de uma forma que usuários com deficiência visual parcial ou total utilizem-se da WEB. Mas ainda há a necessidade de plataformas que possam resolver por inteiro o problema de acessibilidade do usuário, já que em muitos casos, não há apenas uma ferramenta que pode auxiliar a utilização do sistema. 

% Falar sobre o estudo realizado com usuários surdos na Dinamarca

Segundo \citeonline{deaf_people}, mais de um fator pode influenciar na acessibilidade por usuários com deficiência auditiva. Pela pesquisa feita em 2015, \citeonline{deaf_people} cita que há entre os usuários com inabilidade auditiva, problemas de compreensão de textos extensos, além da produtividade interpretativa também ser afetada pela complexidade vocabular utilizada. 

Nesta pesquisa, foram levantadas diversas hipóteses como o papel da imagem como facilitador de compreensão de texto, tempo gasto pelos usuários na completude de uma tarefa é maior em \textit{websites} com poucas imagens e usuários com deficiência auditiva releem mais texto do que os demais usuários. Os resultados demonstram em linhas gerais que as imagens por so só não alteram o desempenho do usuário com inabilidades auditivas, mas que necessidades específicas sim podem a vir alterar o desempenho.

Dentre as análises feitas, em um website com textos muito extensos, usuários com deficiência auditiva tiveram um desempenho inferior ao de uma pessoa com audição normal, demorando mais de 300\% do tempo para completar uma tarefa. 

Para confirmar a hipótese de que um dicionário de palavras em uma linguagem de sinais poderia trazer benefícios ao usuário, a pesquisa montou um grupo de pessoas afim de avaliar esta hipótese. De acordo com classificações qualitativas (medição da dificuldade para completar tarefas), e quantitativas, a pesquisa chegou a conclusão que o usuário sim, encontra mais facilmente informações dentro do \textit{website}, porém o tempo de execução da tarefa não modifica-se tanto, pelo trabalho a mais de se encontrar a palavra no dicionário de Sinais. Vale ressaltar, que a pesquisa levanta que este problema é meramente técnico, e refere-se a um aumento de tempo pela execução de uma tarefa a mais, incluindo-se o tempo de carregamento e fechamento da imagem. Como os tempos com e sem dicionário de sinais foi praticamente o mesmo, considerando-se a consulta como uma tarefa a mais, conclui-se que o tempo de execução da tarefa também diminui consideravelmente.

% Falar sobre os itens abordados no bang do Social4All

O Social4All \cite{social_all} é uma aplicação que busca acessibilizar a WEB para diferentes tipos de pessoas. Seguindo o guia de desenvolvimento da WCAG, o Social4All indica os problemas de acessibilidade encontrados na página, assim como a importância de que este erro seja corrigido, e uma amostra do website de como a página fica seguindo o guia da WCAG. Além das adaptações automáticas, seguindo o Guia de Acessibilidade, o aplicativo também autoriza o usuário a inserir novos problemas encontrados na página, separando a experiência do usuário por perfis. Assim, quando o usuário utiliza-se do sistema, as alterações citadas se tornarão parte das alterações automáticas.

Através de um \textit{gateway} WEB, a aplicação recupera qualquer website e adapta o código original, inserindo padrões definidos no Guia de acessibilidade. O aplicativo também possui um sistema de análise de performance do usuário, que mede o quão vantajosas foram as mudnaças para o usuário, e também mede o nível de acessibilidade já implementado pelo website.

\chapter{Objetivos e Hipótese}

Como há ainda na literatura poucos estudos sobre acessibilidade no meio digital para usuários com inabilidades, o objetivo central do projeto referencia-se ao estudo de formas acessíveis ao usuário deste tipo, com foco nas inabilidades visuais e auditivas. Criar e utilizar-se de ferramentas existentes e que já atuam em questões de acessibilidade no âmbito geral para facilitar o uso destes usuários no meio WEB. 
Ainda dentro do objetivo, temos a proposta de hipótese de que não existe alguma maneira universal de acessibilidade para um website. 


\chapter{Ferramentas de Acessibilidade para Usuários com Deficiência Visual e Auditiva}

O termo acessibilidade não se refere apenas ao meio digital. Diversas ferramentas auxiliam as pessoas em suas tarefas diárias, facilitanto suas tarefas apesar de suas inabilidades. O uso destas ferramentas pode ser transplantado para o meio digital. É claro e evidente que nem todas as ferramentas podem ser utilizadas. Porém, em alguns casos, a computação pode e deve agir com estas ferramentas auxiliares, já que não há um meio comum a tecnologia que possa agir no sentido de acessibilidade por parte do usuário. 

\section{Linguagens de Sinais}



\section{Simplifcação de Texto utilizando Processamento de Linguagem Natural}


\chapter{Padronização WEB}

Desde o inicio da internet, a padronização de mostragem de páginas na rede foi sofrendo alterações. Com o passar dos anos, os textos foram sendo substituidos por imagens e os layouts transformados para se tornarem mais agradáveis e acessiveis. O principal motivo para mudanças foi mudar a experiência do usuário de forma com que as tarefas pudessem ser executadas de maneira mais simples e prática. 

%metodo de pesquisa
 
\section{Coesão textual e visual}

\section{Padrões HTML5}

O \textit{Hyper Text Marking Language} (HTML) é uma linguagem de programação padronizada utilizada para a construção de websites. Desde o ínicio da utilização massiva da internet, a padronização de conteúdo a ser mostrado dentro do HTML foi uma preocupação por parte dos desenvolvedores. Com as ferramentas de pesquisa, essa demanda por padronização se mostrou ainda mais necessária.

Como o padrão foi desenhado em uma época diferente, os propósitos de uma página WEB já não eram mais os mesmos, portanto foi visto a necessidade de mudar as marcações cobertas pela linguagem. Utilizando-se de conceitos previamente desenvolvidos, como \textit{tabless} e \textit{navigation}, permitindo ao programador reproduzir explicitamente as marcações de seções e navegações, respectivamente. 

Mecanismos de busca visam procurar dentro dos websites por informações específicas. Preço de produtos, notícias, informações de redes sociais. A gama de funcionalidades e tipo de informações disponíveis em um website aumentou, e a padronização quanto a localidade destas informações se mostrou necessária. 

Como as aplicações já fazem o uso desta padronização visando o agrupamento destas informações, uma aplicação que vise facilitar o uso da Internet por pessoas com inabilidades específicas pode se utilizar destes padrões. 

Duas padronizações particularmente se mostram interessantes. A navegação do usuário, geralmente realizada por ancoras (\textit{links}) dentro do website agora se mostra presente sempre dentro da marcação <nav>. Assim, ao mostrar as opções de navegação e redirecionamento dentro da página, a aplicação pode procurar imediatamente por conteúdos existentes na marcação <nav>, e mostra-las ao usuário, seja em uma forma visual diferente (aumentando as letras, para usuários com dificuldades visuais), ou até mesmo formando alguma maneira de reproduzir em alguma mídia o seu conteúdo, e posteriormente redirecionar o usuário através de comandos simples, diferentes de comandos padrões de usuários (um clique, por exemplo).

<Insira aqui alguma imagem sobre NAV>

A outra diz respeito à forma com que o conteúdo principal da página é mostrado. Como dito anteriormente, o conceito de \textit{tabless} surgiu antes mesmo do HTML5. Esta prática iniciou-se pois uma diagramação simples das páginas se utilizavam basicamente de tabelas. Porém, o objetivo inicial de uma tabela não condiz com um desenho de uma página por inteiro. Assim, dentro do guia de desenvolvimento padrão de uma página WEB, era inaceitável a utilização de tabelas para a diagramação do \textit{layout} de uma página, sendo indicável a utilização de divisões, ou <div> dentro do \textit{layout}. 

Mas este conceito ainda era insuficiente, mesmo para necessidades básicas de informações no HTML. O próprio exemplo do menu de opções de navegação já demonstra isto. Outro exemplo da ineficiência de utilizar-se apenas divisões é a de páginas com conteúdo escrito denso, ou mesmo páginas com diferentes tipos de conteúdo. 

Páginas de jornais, portais de noticias são exemplos de utulização massiva da marcação \textit{article}. A ideia é manter o conteúdo principal escrito desta página dentro desta marcação.

Outro exemplo refere-se a páginas que possuem várias seções de conteudo diferentes. Muitas páginas procuram dar a experiência do usuário de uma forma corrida e extensa, em uma mesma página, porém, com muitas funcionalidades e conteúdos diferentes. Dai a utilizaçao da marcação \textit{section}, que separa por seções as partes de uma página web.

Com estas divisões bem feitas, facilita-se o trabalho de mecanismos de buscas de conteúdo dentro da página WEB. Os padrões e as indicações feitas pela \citeonline{w3c} vão muito além destas aqui citadas. Porém, pouco acrescentam nas questões específicas de detecção de conteúdo para eventuais mudanças para usuários com inabilidades.

\chapter{Acessibilidade para Usuários com Deficiência Visual}

\section{Uso do \textit{Text to Speech}}

\section{Uma seção secundária}


\subsection{Uma seção terciária}


\subsubsection{Uma seção quartenária}


\subsubsubsection{Uma seção quinária}


\subsubsubsection{Outra seção quinária}

\subsubsubsection{Mais uma seção quinária}


\subsubsection{Outra seção quartenária}


\subsubsection{Mais uma seção quartenária}


\subsection{Outra seção terciária}


\subsection{Mais uma seção terciária}


\section{Outra seção secundária}


\section{Mais uma seção secundária}


\chapter{Mais uma outra seção primária}


\section{Uma seção secundária}


\subsection{Uma seção terciária}



\subsection{Outra seção terciária}

\subsection{Mais uma seção terciária}



\section{Outra seção secundária}


\section{Mais uma seção secundária}

Texto de exemplo, texto de exemplo, texto de exemplo, texto de exemplo, texto de exemplo, texto de exemplo, texto de exemplo, texto de exemplo, texto de exemplo, texto de exemplo, texto de exemplo, texto de exemplo, texto de exemplo, texto de exemplo, texto de exemplo, texto de exemplo, texto de exemplo, texto de exemplo, texto de exemplo, texto de exemplo, texto de exemplo, texto de exemplo, texto de exemplo.

Texto de exemplo, texto de exemplo, texto de exemplo, texto de exemplo, texto de exemplo, texto de exemplo, texto de exemplo, texto de exemplo, texto de exemplo, texto de exemplo, texto de exemplo, texto de exemplo, texto de exemplo, texto de exemplo, texto de exemplo, texto de exemplo, texto de exemplo, texto de exemplo, texto de exemplo, texto de exemplo, texto de exemplo, texto de exemplo, texto de exemplo.

\chapter{Conclusão}

Texto de exemplo, texto de exemplo, texto de exemplo, texto de exemplo, texto de exemplo, texto de exemplo, texto de exemplo, texto de exemplo, texto de exemplo, texto de exemplo, texto de exemplo, texto de exemplo, texto de exemplo, texto de exemplo, texto de exemplo, texto de exemplo, texto de exemplo, texto de exemplo, texto de exemplo, texto de exemplo, texto de exemplo, texto de exemplo, texto de exemplo.

Texto de exemplo, texto de exemplo, texto de exemplo, texto de exemplo, texto de exemplo, texto de exemplo, texto de exemplo, texto de exemplo, texto de exemplo, texto de exemplo, texto de exemplo, texto de exemplo, texto de exemplo, texto de exemplo, texto de exemplo, texto de exemplo, texto de exemplo, texto de exemplo, texto de exemplo, texto de exemplo, texto de exemplo, texto de exemplo.

\section{Uma seção secundária}

Texto de exemplo, texto de exemplo, texto de exemplo, texto de exemplo, texto de exemplo, texto de exemplo, texto de exemplo, texto de exemplo, texto de exemplo, texto de exemplo, texto de exemplo, texto de exemplo, texto de exemplo, texto de exemplo, texto de exemplo, texto de exemplo, texto de exemplo, texto de exemplo, texto de exemplo, texto de exemplo, texto de exemplo, texto de exemplo, texto de exemplo.

Texto de exemplo, texto de exemplo, texto de exemplo, texto de exemplo, texto de exemplo, texto de exemplo, texto de exemplo, texto de exemplo, texto de exemplo, texto de exemplo, texto de exemplo, texto de exemplo, texto de exemplo, texto de exemplo, texto de exemplo, texto de exemplo, texto de exemplo, texto de exemplo, texto de exemplo, texto de exemplo, texto de exemplo, texto de exemplo, texto de exemplo.

\subsection{Uma seção terciária}

Texto de exemplo, texto de exemplo, texto de exemplo, texto de exemplo, texto de exemplo, texto de exemplo, texto de exemplo, texto de exemplo, texto de exemplo, texto de exemplo, texto de exemplo, texto de exemplo, texto de exemplo, texto de exemplo, texto de exemplo, texto de exemplo, texto de exemplo, texto de exemplo, texto de exemplo, texto de exemplo, texto de exemplo, texto de exemplo, texto de exemplo.

\subsection{Outra seção terciária}

Texto de exemplo, texto de exemplo, texto de exemplo, texto de exemplo, texto de exemplo, texto de exemplo, texto de exemplo, texto de exemplo, texto de exemplo, texto de exemplo, texto de exemplo, texto de exemplo, texto de exemplo, texto de exemplo, texto de exemplo, texto de exemplo, texto de exemplo, texto de exemplo, texto de exemplo, texto de exemplo, texto de exemplo, texto de exemplo, texto de exemplo.

\subsection{Mais uma seção terciária}

Texto de exemplo, texto de exemplo, texto de exemplo, texto de exemplo, texto de exemplo, texto de exemplo, texto de exemplo, texto de exemplo, texto de exemplo, texto de exemplo, texto de exemplo, texto de exemplo, texto de exemplo, texto de exemplo, texto de exemplo, texto de exemplo, texto de exemplo, texto de exemplo, texto de exemplo, texto de exemplo, texto de exemplo, texto de exemplo, texto de exemplo.

Texto de exemplo, texto de exemplo, texto de exemplo, texto de exemplo, texto de exemplo, texto de exemplo, texto de exemplo, texto de exemplo, texto de exemplo, texto de exemplo, texto de exemplo, texto de exemplo, texto de exemplo, texto de exemplo, texto de exemplo, texto de exemplo, texto de exemplo, texto de exemplo, texto de exemplo, texto de exemplo, texto de exemplo, texto de exemplo, texto de exemplo.

\section{Outra seção secundária}

Texto de exemplo, texto de exemplo, texto de exemplo, texto de exemplo, texto de exemplo, texto de exemplo, texto de exemplo, texto de exemplo, texto de exemplo, texto de exemplo, texto de exemplo, texto de exemplo, texto de exemplo, texto de exemplo, texto de exemplo, texto de exemplo, texto de exemplo, texto de exemplo, texto de exemplo, texto de exemplo, texto de exemplo, texto de exemplo, texto de exemplo.

Texto de exemplo, texto de exemplo, texto de exemplo, texto de exemplo, texto de exemplo, texto de exemplo, texto de exemplo, texto de exemplo, texto de exemplo, texto de exemplo, texto de exemplo, texto de exemplo, texto de exemplo, texto de exemplo, texto de exemplo, texto de exemplo, texto de exemplo, texto de exemplo, texto de exemplo, texto de exemplo, texto de exemplo, texto de exemplo, texto de exemplo.

Texto de exemplo, texto de exemplo, texto de exemplo, texto de exemplo, texto de exemplo, texto de exemplo, texto de exemplo, texto de exemplo, texto de exemplo, texto de exemplo, texto de exemplo, texto de exemplo, texto de exemplo, texto de exemplo, texto de exemplo, texto de exemplo, texto de exemplo, texto de exemplo, texto de exemplo, texto de exemplo, texto de exemplo, texto de exemplo, texto de exemplo.

\section{Mais uma seção secundária}

Texto de exemplo, texto de exemplo, texto de exemplo, texto de exemplo, texto de exemplo, texto de exemplo, texto de exemplo, texto de exemplo, texto de exemplo, texto de exemplo, texto de exemplo, texto de exemplo, texto de exemplo, texto de exemplo, texto de exemplo, texto de exemplo, texto de exemplo, texto de exemplo, texto de exemplo, texto de exemplo, texto de exemplo, texto de exemplo, texto de exemplo.

Texto de exemplo, texto de exemplo, texto de exemplo, texto de exemplo, texto de exemplo, texto de exemplo, texto de exemplo, texto de exemplo, texto de exemplo, texto de exemplo, texto de exemplo, texto de exemplo, texto de exemplo, texto de exemplo, texto de exemplo, texto de exemplo, texto de exemplo, texto de exemplo, texto de exemplo, texto de exemplo, texto de exemplo, texto de exemplo, texto de exemplo.

% ----------------------------------------------------------
% ELEMENTOS PÓS-TEXTUAIS
% ----------------------------------------------------------
\postextual
% ----------------------------------------------------------

% ----------------------------------------------------------
% Referências bibliográficas
% ----------------------------------------------------------
\bibliography{referencias}

% ----------------------------------------------------------
% Glossário
% ----------------------------------------------------------
%
% Consulte o manual da classe abntex2 para orientações sobre o glossário.
%
%\glossary

% ----------------------------------------------------------
% Apêndices
% ----------------------------------------------------------

% ---
% Inicia os apêndices
% ---
\begin{apendicesenv}

% Imprime uma página indicando o início dos apêndices
%\partapendices

%-------------------------------------------------------------------------
% Informações sobre ``apêndice''
%
% Para todos os captions/(títulos) (de seções, subseções, tabelas, 
% ilustrações, etc):
%     - em maiúscula apenas a primeira letra da sentença (do título), 
%       exceto nomes próprios, geográficos, institucionais ou Programas ou
%       Projetos ou siglas, os quais podem ter letras em maiúscula também.
%
% Todas  as tabelas, ilustrações (figuras, quadros, gráficos, etc. ), 
% anexos, apêndices devem obrigatoriamente ser citados no texto.
%      - a citação deve vir sempre antes da primeira vez em que a tabela, 
%        ilustração, etc., aparecer pela primeira vez.
%
%-------------------------------------------------------------------------
\chapter{Exemplo de apêndice}

Texto de exemplo, texto de exemplo, texto de exemplo, texto de exemplo, texto de exemplo, texto de exemplo, texto de exemplo, texto de exemplo, texto de exemplo, texto de exemplo, texto de exemplo, texto de exemplo, texto de exemplo, texto de exemplo, texto de exemplo, texto de exemplo, texto de exemplo, texto de exemplo, texto de exemplo.

\section*{1 Exemplo de seção de apêndice não apresentada no sumário}

Texto de exemplo, texto de exemplo, texto de exemplo, texto de exemplo, texto de exemplo, texto de exemplo, texto de exemplo, texto de exemplo, texto de exemplo, texto de exemplo, texto de exemplo, texto de exemplo, texto de exemplo, texto de exemplo, texto de exemplo, texto de exemplo, texto de exemplo, texto de exemplo, texto de exemplo.

\subsection*{1.1 Exemplo de subseção de apêndice não apresentada no sumário}

Texto de exemplo, texto de exemplo, texto de exemplo, texto de exemplo, texto de exemplo, texto de exemplo, texto de exemplo, texto de exemplo, texto de exemplo, texto de exemplo, texto de exemplo, texto de exemplo, texto de exemplo, texto de exemplo, texto de exemplo, texto de exemplo, texto de exemplo, texto de exemplo, texto de exemplo.

\subsection*{1.2 Exemplo de subseção de apêndice não apresentada no sumário}

Texto de exemplo, texto de exemplo, texto de exemplo, texto de exemplo, texto de exemplo, texto de exemplo, texto de exemplo, texto de exemplo, texto de exemplo, texto de exemplo, texto de exemplo, texto de exemplo, texto de exemplo, texto de exemplo, texto de exemplo, texto de exemplo, texto de exemplo, texto de exemplo, texto de exemplo.

\subsection*{1.3 Exemplo de subseção de apêndice não apresentada no sumário}

Texto de exemplo, texto de exemplo, texto de exemplo, texto de exemplo, texto de exemplo, texto de exemplo, texto de exemplo, texto de exemplo, texto de exemplo, texto de exemplo, texto de exemplo, texto de exemplo, texto de exemplo, texto de exemplo, texto de exemplo, texto de exemplo, texto de exemplo, texto de exemplo, texto de exemplo.

\section*{2 Exemplo de seção de apêndice não apresentada no sumário}

Texto de exemplo, texto de exemplo, texto de exemplo, texto de exemplo, texto de exemplo, texto de exemplo, texto de exemplo, texto de exemplo, texto de exemplo, texto de exemplo, texto de exemplo, texto de exemplo, texto de exemplo, texto de exemplo, texto de exemplo, texto de exemplo, texto de exemplo, texto de exemplo, texto de exemplo.

\section*{3 Exemplo de seção de apêndice não apresentada no sumário}

Texto de exemplo, texto de exemplo, texto de exemplo, texto de exemplo, texto de exemplo, texto de exemplo, texto de exemplo, texto de exemplo, texto de exemplo, texto de exemplo, texto de exemplo, texto de exemplo, texto de exemplo, texto de exemplo, texto de exemplo, texto de exemplo, texto de exemplo, texto de exemplo, texto de exemplo.


\chapter{Exemplo de apêndice}

Texto de exemplo, texto de exemplo, texto de exemplo, texto de exemplo, texto de exemplo, texto de exemplo, texto de exemplo, texto de exemplo, texto de exemplo, texto de exemplo, texto de exemplo, texto de exemplo, texto de exemplo, texto de exemplo, texto de exemplo, texto de exemplo, texto de exemplo, texto de exemplo, texto de exemplo.

\section*{1 Exemplo de seção de apêndice não apresentada no sumário}

Texto de exemplo, texto de exemplo, texto de exemplo, texto de exemplo, texto de exemplo, texto de exemplo, texto de exemplo, texto de exemplo, texto de exemplo, texto de exemplo, texto de exemplo, texto de exemplo, texto de exemplo, texto de exemplo, texto de exemplo, texto de exemplo, texto de exemplo, texto de exemplo, texto de exemplo.

\subsection*{1.1 Exemplo de subseção de apêndice não apresentada no sumário}

Texto de exemplo, texto de exemplo, texto de exemplo, texto de exemplo, texto de exemplo, texto de exemplo, texto de exemplo, texto de exemplo, texto de exemplo, texto de exemplo, texto de exemplo, texto de exemplo, texto de exemplo, texto de exemplo, texto de exemplo, texto de exemplo, texto de exemplo, texto de exemplo, texto de exemplo.

\subsection*{1.2 Exemplo de subseção de apêndice não apresentada no sumário}

Texto de exemplo, texto de exemplo, texto de exemplo, texto de exemplo, texto de exemplo, texto de exemplo, texto de exemplo, texto de exemplo, texto de exemplo, texto de exemplo, texto de exemplo, texto de exemplo, texto de exemplo, texto de exemplo, texto de exemplo, texto de exemplo, texto de exemplo, texto de exemplo, texto de exemplo.

\subsection*{1.3 Exemplo de subseção de apêndice não apresentada no sumário}

Texto de exemplo, texto de exemplo, texto de exemplo, texto de exemplo, texto de exemplo, texto de exemplo, texto de exemplo, texto de exemplo, texto de exemplo, texto de exemplo, texto de exemplo, texto de exemplo, texto de exemplo, texto de exemplo, texto de exemplo, texto de exemplo, texto de exemplo, texto de exemplo, texto de exemplo.

\section*{2 Exemplo de seção de apêndice não apresentada no sumário}

Texto de exemplo, texto de exemplo, texto de exemplo, texto de exemplo, texto de exemplo, texto de exemplo, texto de exemplo, texto de exemplo, texto de exemplo, texto de exemplo, texto de exemplo, texto de exemplo, texto de exemplo, texto de exemplo, texto de exemplo, texto de exemplo, texto de exemplo, texto de exemplo, texto de exemplo.

\section*{3 Exemplo de seção de apêndice não apresentada no sumário}

Texto de exemplo, texto de exemplo, texto de exemplo, texto de exemplo, texto de exemplo, texto de exemplo, texto de exemplo, texto de exemplo, texto de exemplo, texto de exemplo, texto de exemplo, texto de exemplo, texto de exemplo, texto de exemplo, texto de exemplo, texto de exemplo, texto de exemplo, texto de exemplo, texto de exemplo.


\chapter{Exemplo de apêndice}

Texto de exemplo, texto de exemplo, texto de exemplo, texto de exemplo, texto de exemplo, texto de exemplo, texto de exemplo, texto de exemplo, texto de exemplo, texto de exemplo, texto de exemplo, texto de exemplo, texto de exemplo, texto de exemplo, texto de exemplo, texto de exemplo, texto de exemplo, texto de exemplo, texto de exemplo.

\section*{1 Exemplo de seção de apêndice não apresentada no sumário}

Texto de exemplo, texto de exemplo, texto de exemplo, texto de exemplo, texto de exemplo, texto de exemplo, texto de exemplo, texto de exemplo, texto de exemplo, texto de exemplo, texto de exemplo, texto de exemplo, texto de exemplo, texto de exemplo, texto de exemplo, texto de exemplo, texto de exemplo, texto de exemplo, texto de exemplo.

\subsection*{1.1 Exemplo de subseção de apêndice não apresentada no sumário}

Texto de exemplo, texto de exemplo, texto de exemplo, texto de exemplo, texto de exemplo, texto de exemplo, texto de exemplo, texto de exemplo, texto de exemplo, texto de exemplo, texto de exemplo, texto de exemplo, texto de exemplo, texto de exemplo, texto de exemplo, texto de exemplo, texto de exemplo, texto de exemplo, texto de exemplo.

\subsection*{1.2 Exemplo de subseção de apêndice não apresentada no sumário}

Texto de exemplo, texto de exemplo, texto de exemplo, texto de exemplo, texto de exemplo, texto de exemplo, texto de exemplo, texto de exemplo, texto de exemplo, texto de exemplo, texto de exemplo, texto de exemplo, texto de exemplo, texto de exemplo, texto de exemplo, texto de exemplo, texto de exemplo, texto de exemplo, texto de exemplo.

\subsection*{1.3 Exemplo de subseção de apêndice não apresentada no sumário}

Texto de exemplo, texto de exemplo, texto de exemplo, texto de exemplo, texto de exemplo, texto de exemplo, texto de exemplo, texto de exemplo, texto de exemplo, texto de exemplo, texto de exemplo, texto de exemplo, texto de exemplo, texto de exemplo, texto de exemplo, texto de exemplo, texto de exemplo, texto de exemplo, texto de exemplo.

\section*{2 Exemplo de seção de apêndice não apresentada no sumário}

Texto de exemplo, texto de exemplo, texto de exemplo, texto de exemplo, texto de exemplo, texto de exemplo, texto de exemplo, texto de exemplo, texto de exemplo, texto de exemplo, texto de exemplo, texto de exemplo, texto de exemplo, texto de exemplo, texto de exemplo, texto de exemplo, texto de exemplo, texto de exemplo, texto de exemplo.

\section*{3 Exemplo de seção de apêndice não apresentada no sumário}

Texto de exemplo, texto de exemplo, texto de exemplo, texto de exemplo, texto de exemplo, texto de exemplo, texto de exemplo, texto de exemplo, texto de exemplo, texto de exemplo, texto de exemplo, texto de exemplo, texto de exemplo, texto de exemplo, texto de exemplo, texto de exemplo, texto de exemplo, texto de exemplo, texto de exemplo.

\end{apendicesenv}
% ---


% ----------------------------------------------------------
% Anexos
% ----------------------------------------------------------

% ---
% Inicia os anexos
% ---
\begin{anexosenv}

% Imprime uma página indicando o início dos anexos
%\partanexos


%-------------------------------------------------------------------------
% Informações sobre ``anexo''
%
% Para todos os captions/(títulos) (de seções, subseções, tabelas, 
% ilustrações, etc):
%     - em maiúscula apenas a primeira letra da sentença (do título), 
%       exceto nomes próprios, geográficos, institucionais ou Programas ou
%       Projetos ou siglas, os quais podem ter letras em maiúscula também.
%
% Todas  as tabelas, ilustrações (figuras, quadros, gráficos, etc. ), 
% anexos, apêndices devem obrigatoriamente ser citados no texto.
%      - a citação deve vir sempre antes da primeira vez em que a tabela, 
%        ilustração, etc., aparecer pela primeira vez.
%
%-------------------------------------------------------------------------
\chapter{Resumo das normas}
\label{anexoA}

Considerando a dificuldade para formatar um texto acadêmico sem conhecimento básico do conteúdo da norma NBR 14724 ``Informação e documentação – Trabalhos acadêmicos – Apresentação'', este anexo apresenta um resumo de alguns conceitos dessa norma, conforme publicada em julho de 2011. Sugere-se a leitura completa da norma para garantir que seu documento seja completamente aderente à mesma.

\section*{1 NBR 14724: estrutura e algumas descrições}

A estrutura de uma tese, dissertação ou qualquer outro trabalho acadêmico, deve compreender elementos pré-textuais, elementos textuais e elementos pós-textuais, que aparecem no texto na seguinte ordem:

\subsection*{1.1 Elementos pré-textuais}

\begin{itemize}
	\item Capa (obrigatório)
	\item	Folha de rosto (obrigatório)
	\item	Errata (opcional)
	\item	Folha de aprovação (opcional)
	\item	Dedicatória (opcional)
	\item	Agradecimentos (opcional)
	\item	Epígrafe (opcional)
	\item	Resumo em língua vernácula (obrigatório)
	\item	Resumo em língua estrangeira (obrigatório)
	\item	Listas de ilustrações: lista de figuras, lista de algoritmos, lista de quadros, etc. (opcional)
	\item	Lista de tabelas (opcional)
	\item	Lista de abreviaturas e siglas (opcional)
	\item	Lista de símbolos (opcional)
	\item	Sumário (obrigatório)
\end{itemize}

\subsection*{1.2 Elementos textuais}

\begin{itemize}
	\item	Introdução
	\item	Desenvolvimento
	\item	Conclusão
\end{itemize}

\subsection*{1.3 Elementos pós-textuais}

\begin{itemize}
	\item	Referências (obrigatório)
	\item	Apêndice (opcional)
	\item	Anexo (opcional)
	\item	Glossário (opcional)
\end{itemize}

\section*{2 Definições relacionadas a elementos pré-textuais}

A seguir, são apresentadas algumas definições contidas na norma relacionadas a elementos pré-textuais.

\subsection*{2.1 Capa}

Elemento obrigatório, para proteção externa e sobre o qual se imprimem informações que ajudam na identificação e uso do trabalho, na seguinte ordem:
\begin{enumerate}
	\item Nome completo do autor: responsável intelectual do trabalho.
	\item	Título principal do trabalho: deve ser claro e preciso, identificando o seu conteúdo e possibilitando a indexação e recuperação da informação.
	\item	Subtítulo (se houver): deve ser evidenciada sua subordinação ao título principal, precedido de dois pontos (:).
	\item	Número do volume (obrigatório apenas se houver mais de um volume, de forma que deve constar em cada capa a especificação do respectivo volume).
	\item	Local (cidade) da instituição de apresentação.
	\item	Ano do depósito (entrega).
\end{enumerate}

\subsection*{2.2 Folha de rosto (anverso)}

Os elementos do anverso da folha de rosto devem figurar na seguinte ordem:
\begin{enumerate}
	\item	Nome completo do autor: responsável intelectual do trabalho.
	\item	Título principal do trabalho: deve ser claro e preciso, identificando o seu conteúdo e possibilitando a indexação e recuperação da informação.
	\item	Subtítulo (se houver): deve ser evidenciada sua subordinação ao título principal, precedido de dois pontos (:).
	\item	Número do volume (obrigatório apenas se houver mais de um volume, de forma que deve constar em cada capa a especificação do respectivo volume).
	\item	Natureza (tese, dissertação e outros) e objetivo (aprovação em disciplina, grau pretendido e outros); nome da instituição a que é submetido; área de concentração.
	\item	Nome do orientador e, se houver, do co-orientador.
	\item	Local (cidade) da instituição de apresentação.
	\item	Ano de depósito (entrega).
\end{enumerate}

\subsection*{2.3 Dedicatória e agradecimentos}

Elementos opcionais. Os agradecimentos devem ser dirigidos apenas àqueles que contribuíram de maneira relevante à elaboração do trabalho.

\subsection*{2.4 Resumo na língua vernácula}

Elemento obrigatório, que consiste na apresentação concisa dos pontos relevantes de um texto; constitui-se em uma sequência de frases concisas e objetivas, e não de uma simples enumeração de tópicos, não ultrapassando 500 palavras, seguido, logo abaixo, das palavras representativas do conteúdo do trabalho, isto é, palavras-chave e/ou descritores.

\subsection*{2.5 Resumo em língua estrangeira}

Elemento obrigatório, que consiste em uma versão do resumo em idioma de divulgação internacional (em inglês Abstract, em castelhano Resumen, em francês Résumé, por exemplo). Deve ser seguido das palavras representativas do conteúdo do trabalho, isto é, palavras-chave e/ou descritores, na respectiva língua estrangeira.

\subsection*{2.6 Lista de figuras e lista de tabelas}

Elementos opcionais, elaborados de acordo com a ordem apresentada no texto, com cada item acompanhado do respectivo número da página.

\subsection*{2.7 Lista de abreviaturas e siglas}

Elemento opcional. Consiste na relação alfabética das abreviaturas e siglas usadas no texto, seguidas das palavras ou expressões correspondentes grafadas por extenso.

\subsection*{2.8 Lista de símbolos}

Elemento opcional, elaborado de acordo com a ordem apresentada no texto, com o devido significado.

\subsection*{2.9 Sumário}

Elemento obrigatório, que consiste na enumeração das principais divisões (seções e outras partes do trabalho) dos elementos textuais e pós-textuais, na mesma ordem e grafia em que a matéria nele sucede, acompanhado do respectivo número da página.

\section*{3 Definições relacionadas a elementos textuais}

O autor deve criar quantas seções primárias (também chamadas informalmente de capítulos) desejar para tratar dos seguintes elementos textuais que são obrigatórios: introdução, desenvolvimento e conclusão. Normalmente, existe apenas uma seção primária para a introdução, uma ou mais seções primárias para o desenvolvimento, e apenas uma seção primária para a conclusão.

\section*{4 Definições relacionadas a elementos pós-textuais}

A seguir, são apresentadas algumas definições contidas na norma relacionadas a elementos pós-textuais.

\subsection*{4.1 Apêndice}

Elemento opcional, que consiste em um texto ou documento elaborado pelo próprio autor, a fim de complementar sua argumentação, sem prejuízo da unidade nuclear do trabalho. Um apêndice deve ser identificado por uma letra maiúscula, seguida por um hífen (entre caracteres de espaço), seguido pelo respectivo título. Os apêndices devem ser identificados por letras consecutivas, a partir da letra ``A'' (independentemente dos anexos).

\subsection*{4.2 Anexo}

Elemento opcional, que consiste em um texto ou documento não elaborado pelo autor, a fim de fundamentar, comprovar ou ilustrar a argumentação do autor. Um anexo deve ser identificado por uma letra maiúscula, seguida por um hífen (entre caracteres de espaço), seguido pelo respectivo título. Os anexos devem ser identificados por letras consecutivas, a partir da letra ``A'' (independentemente dos apêndices).

\subsection*{4.3 Glossário}

Elemento opcional, que consiste em uma lista em ordem alfabética de palavras ou de expressões técnicas de uso restrito ou de sentido obscuro, usadas no texto, acompanhadas das respectivas definições.

\section*{5 Formas de apresentação}

A seguir, são apresentadas algumas definições contidas na norma relacionadas a formas de apresentação em geral.

\subsection*{5.1 Formato}

O texto deve estar impresso em papel branco, formato A4 (21,0 cm 29,7 cm), apenas no anverso da folha (ou seja, na ``frente'' da folha), excetuando-se a folha de rosto que deve estar impressa tanto no anverso quanto no verso (com a ficha catalográfica).

\subsection*{5.2 Projeto gráfico}

O projeto gráfico é de responsabilidade do autor.

\subsection*{5.3 Fonte}


Usar sempre cor preta.

Usar sempre tamanho de fonte 12, com as seguintes exceções: tamanho de fonte 10 para citações longas (com mais de três linhas), notas de rodapé, legendas de ilustração e de tabela, fontes de ilustração e de tabela, números de página; e tamanho de fonte maiores para títulos de seção (conforme apresentado na seção 6.1 a seguir).

\subsection*{5.4 Margens}

Todas as folhas devem apresentar margens esquerda e superior de 3 cm; e margens direita e inferior de 2 cm, considerando impressão apenas no anverso (ou seja, apenas na ``frente''). 

Se a impressão precisar, por algumo motivo especial, ser realizada em anverso e verso (ou seja, em frente e verso), neste caso, há que se configurar as margens de forma diferente, conforme detalhes da norma ABNT, além de outros detalhes de configuração; por isso solicita-se não realizar impressão em frente e verso.

\subsection*{5.5 Espaçamento entre linhas}

Usar sempre espaçamento entre linhas de 1,5 linhas, com as seguintes exceções: espaçamento entre linhas ``simples'' para citações longas (com mais de três linhas), notas de rodapé, referências, resumos (em vernáculo e em língua estrangeira), legendas de ilustração e de tabela, fontes de ilustração e de tabela, ficha catalográfica, natureza do trabalho, grau pretendido, nome da instituição a que é submetido, e área de concentração; e espaçamento entre linhas ``duplo'' para equações e fórmulas e para separação das referências entre si.

Os títulos das seções devem começar na margem superior da folha separados do texto que os sucede por um espaço em branco de 1,5 e, da mesma forma, os títulos das subseções devem ser separados do texto que os precede, ou que os sucede, por um espaço em branco de 1,5.

\subsection*{5.6 Numeração das seções}

O indicativo numérico de uma seção precede seu título, alinhado à esquerda, separado por um espaço de caractere. Nos títulos sem indicativo numérico, como lista de ilustrações, sumário, resumo, referências e outros, devem ser centralizados.

Para evidenciar a sistematização do conteúdo do trabalho, deve-se adotar a numeração progressiva para as seções do texto. Os títulos das seções primárias (chamadas informalmente de capítulos), por serem as principais divisões do texto, devem iniciar em folha distinta. Títulos das seções e subseções devem ser destacados gradativamente, usando-se os recursos de negrito, itálico ou grifo e redondo, caixa alta ou versal.

\subsection*{5.7 Paginação}

Todas as folhas do trabalho, a partir da folha de rosto (desconsiderando a capa, mas considerando a ficha catalográfica), devem ser contadas sequencialmente, mas não numeradas. A numeração é colocada, a partir da primeira folha da dos elementos textuais (ou seja, a partir da ``Introdução''), em algarismos arábicos, no canto superior direito da folha, a 2 cm da borda superior, ficando o último algarismo a 2 cm da borda direita da folha.

Havendo apêndices e/ou anexos, suas folhas devem ser numeradas de maneira contínua e sua paginação deve dar seguimento à do texto principal, em algarismos arábicos.

No caso de o trabalho ser constituído de mais de um volume, deve-se manter uma única sequência de numeração das folhas, do primeiro ao último volume.

\subsection*{5.8 Equações e fórmulas}

Equações e fórmulas devem aparecer destacadas no texto, para facilitar sua leitura. 

Se as equações e fórmulas forem apresentadas na sequência normal do texto (ou seja, dentro do próprio parágrafo normal de texto), é permitido usar um espaçamento entre linhas duplo para comportar seus elementos (ou seja, expoentes, índices e outros). 

Se as equações e fórmulas forem apresentadas fora do parágrafo, então elas devem ser centralizadas e, se necessário, devem ser numeradas. Quando fragmentadas em mais de uma linha, por falta de espaço, devem ser interrompidas antes do sinal de igualdade ou depois dos sinais de adição, subtração, multiplicação e divisão.

\subsection*{5.9 Ilustrações}

Cada tipo de ilustração (tais como figura, gráfico, algoritmo, fotografia, quadro, esquema, desenhos, esquemas, fluxogramas, mapa, organograma, planta, retrato, entre outros) tem numeração independente e consecutiva. 

Inserir a ilustração o mais próximo possível do parágrafo em que ela é citada pela primeira vez no texto; nunca inserir uma ilustração antes de ela ser citada pela primeira vez no texto. Toda ilustração inserida no trabalho deve ser citada pelo menos uma vez no texto.

Qualquer que seja o tipo da ilustração, ela deve obrigatoriamente ter uma identificação (ou seja, um título), que deve aparecer sempre na parte superior da ilustração, precedida pela palavra que identifica seu tipo, por exemplo ``Figura'', seguida de seu número de ordem de ocorrência no texto em algarismo arábico, e de um hífen entre caracteres de espaço (`` – ''), em fonte com tamanho 12, sem negrito, sem itálico, com apenas a primeira letra da sentença maiúscula, sem ponto final, e em espaçamento simples. Exemplo: ``Figura 1 – Título da ilustração''.

Para toda ilustração, deve ser apresentada também obrigatoriamente sua fonte (mesmo quando a fonte é o próprio autor do trabalho). A fonte deve apresentada na parte inferior da ilustração e ser informada no seguinte formato: palavra ``Fonte'', seguida pelo caractere dois pontos ``:'', seguido por um caractere de espaço, seguido pela citação de onde a ilustração foi obtida (conforme regras de citação da norma ABNT) ou seguido pelo nome completo do autor do trabalho, por uma vírgula e pelo ano de elaboração do trabalho (caso a ilustração seja de elaboração do próprio autor), em fonte com tamanho 10, sem negrito, sem itálico, sem ponto final, e em espaçamento simples. Exemplo 1 (quando se trata de fonte externa): ``Fonte: citação conforme norma ABNT''; Exemplo 2 (quando se trata do próprio autor do trabalho): ``Fonte: Nome Completo, Ano''.

Para referenciar uma ilustração (por exemplo, do tipo ``figura'') no texto, há duas formas: \textit{(i)} se a referência à figura fizer parte do texto, mesmo que dentro de parênteses, use a palavra ``figura'' com todas as letras em minúsculo, por exemplo – ``A figura 5 apresenta um exemplo de (...)'' ou ``(...) esses dados já foram apresentados na seção anterior (ver figura 5)''; \textit{(ii)} se a referência à figura estiver completamente isolada do texto, dentro de parênteses, use a palavra ``Figura'' com a inicial em maiúsculo, por exemplo ``(...) para um entendimento mais claro, essas informações estão apresentadas graficamente (Figura 5)''.

\subsection*{5.10 Tabelas}

As tabelas têm numeração independente e consecutiva das ilustrações.

Inserir a tabela o mais próximo possível do parágrafo em que ela é citada pela primeira vez no texto; nunca inserir uma tabela antes de ela ser citada pela primeira vez no texto. Toda tabela inserida no trabalho deve ser citada pelo menos uma vez no texto.

Toda tabela deve obrigatoriamente ter uma identificação (ou seja, um título), que deve aparece na parte superior, precedida pela palavra ``Tabela'', seguida de seu número de ordem de ocorrência no texto em algarismo arábico, e de um hífen entre caracteres de espaço (`` – ''), em fonte com tamanho 12, sem negrito, sem itálico, com apenas a primeira letra da sentença maiúscula, sem ponto final, e em espaçamento simples. Exemplo: ``Tabela 1 – Título da tabela''.

Para toda tabela, deve ser apresentada também obrigatoriamente sua fonte (mesmo quando a fonte é o próprio autor do trabalho). A fonte deve apresentada na parte inferior da tabela e ser informada no seguinte formato: palavra ``Fonte'', seguida pelo caractere dois pontos ``:'', seguido por um caractere de espaço, seguido pela citação de onde a fonte foi obtida (conforme regras de citação da norma ABNT) ou seguido pelo nome completo do autor do trabalho, por uma vírgula e pelo ano de elaboração do trabalho (caso a fonte seja de elaboração do próprio autor), em fonte com tamanho 10, sem negrito, sem itálico, sem ponto final, e em espaçamento simples. Exemplo 1 (quando se trata de fonte externa): ``Fonte: citação conforme norma ABNT''; Exemplo 2 (quando se trata do próprio autor do trabalho): ``Fonte: Nome Completo, Ano''.

Usar traços horizontais apenas para delimitar o cabeçalho da tabela e o início e o fim da tabela. Não usar traços horizontais para separar cada linha de conteúdo da tabela e também não usar traços verticais para separar cada coluna de conteúdo da tabela. 

Se a tabela não couber em uma folha, ela deve ser continuada nas folhas seguintes. Nesse caso, a tabela não deve ser delimitada por traço horizontal na parte inferior nas primeiras folhas (mas sim apenas na última folha em que ela realmente é finalizada), e a legenda e o cabeçalho da tabela devem ser repetidos nas folhas seguintes. Além disso, as folhas devem ter as seguintes indicações: ``continua'' (no fim das primeiras folhas); ``continuação'' (no início das folhas intermediárias, se houver) e ``conclusão'' (no início da última folha).

Para referenciar uma tabela no texto, há duas formas: \textit{(i)} se a referência à tabela fizer parte do texto, mesmo que dentro de parênteses, use a palavra ``tabela'' com todas as letras em minúsculo, por exemplo – ``A tabela 5 apresenta um exemplo de (...)'' ou ``(...) esses dados já foram apresentados na seção anterior (ver tabela 5)''; \textit{(ii)} se a referência à tabela estiver completamente isolada do texto, dentro de parênteses, use a palavra ``Tabela'' com a inicial em maiúsculo, por exemplo ``(...) para um entendimento mais claro, essas informações estão apresentadas graficamente (Tabela 5)''.

Não confundir ``tabela'' com ``quadro''. Uma tabela deve ter dados numéricos como informação central. Outros tipos de organização de informações devem ser apresentados em quadros, que é um dos tipos de ilustração. A formatação de um quadro é muito parecida a de uma tabela, porém todos os traços horizontais e verticais devem ser apresentados.

\section*{6 Outras normas}

\subsection*{6.1 Seções}

As seções primárias são as principais divisões do texto, denominadas informalmente de ``capítulos''. As seções primárias podem ser divididas em seções secundárias; as secundárias em terciárias, e assim por diante, até a quinta ordem, em formatação distinta. Não é possível dividir o texto mais do que a quinta ordem.

A formatação adotada para este \textit{template} em particular é a seguinte:

\begin{itemize}
	\item Seções primárias: tamanho 16, com negrito.
	\item Seções secundárias: tamanho 15, com negrito.
	\item Seções terciárias: tamanho 14, com negrito.
	\item Seções quartenárias: tamanho 13, sem negrito.
	\item Seções quinárias: tamanho 12, sem negrito.
\end{itemize}

São empregados algarismos arábicos na numeração. O ``indicativo'' de uma seção precede o título ou a primeira palavra do texto, se não houver título, separado por um espaço. O indicativo da seção secundária é constituído pelo indicativo da seção primária que a precede seguido do número que lhe foi atribuído na sequência do assunto e separado por ponto. Repete-se o mesmo processo em relação às demais seções. Na leitura, não se lê os pontos (por exemplo: ``2.1.1'' lê-se ``dois um um'').

Os indicativos devem ser citados no texto de acordo com os seguintes exemplos: (...) na seção 4 (...); (...) no capítulo 2 (...); (...) ver 9.2 (...); (...) em 1.1.2.2 parág. 3º [ou] (...) no 3º parágrafo de 1.1.2.2; (...) (Seção 2.1) (...).

\subsection*{6.2 Referências bibliográficas e citações às referências bibliográficas}

A norma é bastante complexa e extensa em relação às regras de referências bibliográficas (cerca de 19 páginas) e citações às referências bibliográficas, não sendo possível fazer um resumo aqui. Assim, é necessário fazer uma consulta às normas detalhadas.

As referências devem ser apresentadas em ordem alfabética, com as citações no texto obedecendo ao sistema autor-data.Todos os documentos relacionados nas Referências devem ser citados no texto, assim como todas as citações do texto devem constar nas Referências.

\chapter{Exemplo de anexo}

Texto de exemplo, texto de exemplo, texto de exemplo, texto de exemplo, texto de exemplo, texto de exemplo, texto de exemplo, texto de exemplo, texto de exemplo, texto de exemplo, texto de exemplo, texto de exemplo, texto de exemplo, texto de exemplo, texto de exemplo, texto de exemplo, texto de exemplo, texto de exemplo, texto de exemplo.

\section*{1 Exemplo de seção de anexo não apresentada no sumário}

Texto de exemplo, texto de exemplo, texto de exemplo, texto de exemplo, texto de exemplo, texto de exemplo, texto de exemplo, texto de exemplo, texto de exemplo, texto de exemplo, texto de exemplo, texto de exemplo, texto de exemplo, texto de exemplo, texto de exemplo, texto de exemplo, texto de exemplo, texto de exemplo, texto de exemplo.

\subsection*{1.1 Exemplo de subseção de anexo não apresentada no sumário}

Texto de exemplo, texto de exemplo, texto de exemplo, texto de exemplo, texto de exemplo, texto de exemplo, texto de exemplo, texto de exemplo, texto de exemplo, texto de exemplo, texto de exemplo, texto de exemplo, texto de exemplo, texto de exemplo, texto de exemplo, texto de exemplo, texto de exemplo, texto de exemplo, texto de exemplo.

\subsection*{1.2 Exemplo de subseção de anexo não apresentada no sumário}

Texto de exemplo, texto de exemplo, texto de exemplo, texto de exemplo, texto de exemplo, texto de exemplo, texto de exemplo, texto de exemplo, texto de exemplo, texto de exemplo, texto de exemplo, texto de exemplo, texto de exemplo, texto de exemplo, texto de exemplo, texto de exemplo, texto de exemplo, texto de exemplo, texto de exemplo.

\subsection*{1.3 Exemplo de subseção de anexo não apresentada no sumário}

Texto de exemplo, texto de exemplo, texto de exemplo, texto de exemplo, texto de exemplo, texto de exemplo, texto de exemplo, texto de exemplo, texto de exemplo, texto de exemplo, texto de exemplo, texto de exemplo, texto de exemplo, texto de exemplo, texto de exemplo, texto de exemplo, texto de exemplo, texto de exemplo, texto de exemplo.

\section*{2 Exemplo de seção de anexo não apresentada no sumário}

Texto de exemplo, texto de exemplo, texto de exemplo, texto de exemplo, texto de exemplo, texto de exemplo, texto de exemplo, texto de exemplo, texto de exemplo, texto de exemplo, texto de exemplo, texto de exemplo, texto de exemplo, texto de exemplo, texto de exemplo, texto de exemplo, texto de exemplo, texto de exemplo, texto de exemplo.

\section*{3 Exemplo de seção de anexo não apresentada no sumário}

Texto de exemplo, texto de exemplo, texto de exemplo, texto de exemplo, texto de exemplo, texto de exemplo, texto de exemplo, texto de exemplo, texto de exemplo, texto de exemplo, texto de exemplo, texto de exemplo, texto de exemplo, texto de exemplo, texto de exemplo, texto de exemplo, texto de exemplo, texto de exemplo, texto de exemplo.

\chapter{Exemplo de anexo}

Texto de exemplo, texto de exemplo, texto de exemplo, texto de exemplo, texto de exemplo, texto de exemplo, texto de exemplo, texto de exemplo, texto de exemplo, texto de exemplo, texto de exemplo, texto de exemplo, texto de exemplo, texto de exemplo, texto de exemplo, texto de exemplo, texto de exemplo, texto de exemplo, texto de exemplo.

\section*{1 Exemplo de seção de anexo não apresentada no sumário}

Texto de exemplo, texto de exemplo, texto de exemplo, texto de exemplo, texto de exemplo, texto de exemplo, texto de exemplo, texto de exemplo, texto de exemplo, texto de exemplo, texto de exemplo, texto de exemplo, texto de exemplo, texto de exemplo, texto de exemplo, texto de exemplo, texto de exemplo, texto de exemplo, texto de exemplo.

\subsection*{1.1 Exemplo de subseção de anexo não apresentada no sumário}

Texto de exemplo, texto de exemplo, texto de exemplo, texto de exemplo, texto de exemplo, texto de exemplo, texto de exemplo, texto de exemplo, texto de exemplo, texto de exemplo, texto de exemplo, texto de exemplo, texto de exemplo, texto de exemplo, texto de exemplo, texto de exemplo, texto de exemplo, texto de exemplo, texto de exemplo.

\subsection*{1.2 Exemplo de subseção de anexo não apresentada no sumário}

Texto de exemplo, texto de exemplo, texto de exemplo, texto de exemplo, texto de exemplo, texto de exemplo, texto de exemplo, texto de exemplo, texto de exemplo, texto de exemplo, texto de exemplo, texto de exemplo, texto de exemplo, texto de exemplo, texto de exemplo, texto de exemplo, texto de exemplo, texto de exemplo, texto de exemplo.

\subsection*{1.3 Exemplo de subseção de anexo não apresentada no sumário}

Texto de exemplo, texto de exemplo, texto de exemplo, texto de exemplo, texto de exemplo, texto de exemplo, texto de exemplo, texto de exemplo, texto de exemplo, texto de exemplo, texto de exemplo, texto de exemplo, texto de exemplo, texto de exemplo, texto de exemplo, texto de exemplo, texto de exemplo, texto de exemplo, texto de exemplo.

\section*{2 Exemplo de seção de anexo não apresentada no sumário}

Texto de exemplo, texto de exemplo, texto de exemplo, texto de exemplo, texto de exemplo, texto de exemplo, texto de exemplo, texto de exemplo, texto de exemplo, texto de exemplo, texto de exemplo, texto de exemplo, texto de exemplo, texto de exemplo, texto de exemplo, texto de exemplo, texto de exemplo, texto de exemplo, texto de exemplo.

\section*{3 Exemplo de seção de anexo não apresentada no sumário}

Texto de exemplo, texto de exemplo, texto de exemplo, texto de exemplo, texto de exemplo, texto de exemplo, texto de exemplo, texto de exemplo, texto de exemplo, texto de exemplo, texto de exemplo, texto de exemplo, texto de exemplo, texto de exemplo, texto de exemplo, texto de exemplo, texto de exemplo, texto de exemplo, texto de exemplo.

\end{anexosenv}

%---------------------------------------------------------------------
% INDICE REMISSIVO
%---------------------------------------------------------------------
%%%%%MF\phantompart
%%%%%MF\printindex
%---------------------------------------------------------------------

\end{document}
